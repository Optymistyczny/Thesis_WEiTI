%--------------------------------------------------------
% Szablon pracy dyplomowej inżynierskiej i magisterskiej
% Opracowanie: Jakub Stankiewicz
% Politechnika Warszawska, 2024-2025
% 
% Ten szablon jest udostępniony na licencji Creative 
% Commons Attribution-ShareAlike 4.0 International (CC BY-SA 4.0).
% Masz prawo do:
% - Udostępniania: kopiowania i rozpowszechniania materiału na dowolnym nośniku
%   i w dowolnym formacie.
% - Adaptacji: remiksowania, przekształcania i tworzenia nowych materiałów na 
%   podstawie materiału, również do celów komercyjnych.
%
% Na następujących warunkach:
% - Uznanie autorstwa: musisz odpowiednio oznaczyć autorstwo, podać link do licencji
%   oraz wskazać, czy wprowadziłeś zmiany. Możesz to zrobić w dowolny sposób,
%   ale nie w sposób sugerujący, że licencjodawca popiera ciebie lub twoje użycie.
% - Na tych samych warunkach: jeśli remiksujesz, przekształcasz lub tworzysz nowe 
%   materiały na podstawie oryginału, musisz rozpowszechniać swoje dzieło na tej 
%   samej licencji, co oryginał.
% 
% Szczegóły licencji: https://creativecommons.org/licenses/by-sa/4.0/
%
% Zrzeczenie się odpowiedzialności:
% Szablon jest udostępniony "tak jak jest" bez żadnej gwarancji lub zapewnienia
% przydatności do określonego celu. Autor nie ponosi odpowiedzialności za 
% jakiekolwiek problemy wynikające z jego użycia.
%--------------------------------------------------------
%  - Bibliografia: Kamil Zdanowicz, Krzysztof Dobrzyniecki  
%  - Strona tytułowa: Rafał Łysoń
%  - Treść szablonu: Jakub Stankiewicz
%--------------------------------------------------------
%  Sposób kompilacji:
%  pdflatex -> biber -> pdflatex x2
%--------------------------------------------------------
%  MiKTeX-pdfTeX 4.19 (MiKTeX 24.4)
%  Wersja biber: 2.20
%--------------------------------------------------------
% !TEX root = main.tex
\documentclass[
    bindingoffset=0mm,  % Margines na oprawę
    footnoteindent=0mm, % Wcięcie dla przypisów
    hyphenation=true,   % Włączenie/wyłączenie dzielenia wyrazów
    fontsize=11pt,      % Domyślny rozmiar czcionki [11pt]
]{src/wut-thesis}

% Odkomentowanie poniższej linii spowoduje kompilację 
% tylko wybranych plików. Stare odwołania i 
% numeracja zostaną zachowane. Przydatne w przypadku dużych prac.
% \includeonly{tex/0-abstrakt}          % Częściowa kompilacja 
% \includeonly{tex/1-czesc-teoretyczna} % Częściowa kompilacja 
% \includeonly{tex/2-czesc-wykonawcza}  % Częściowa kompilacja 
% \includeonly{tex/3-podsumowanie}      % Częściowa kompilacja 

\graphicspath{{img/}}               % Katalog z ilustracjami
\addbibresource{bibliografia.bib}   % Plik .bib z bibliografią
\hypersetup{
    pdftitle={Temat pracy dyplomowej},
    pdfsubject={Zagadnienia poruszane w pracy},
    pdfauthor={Autor},
    pdfcreator={TeX/LaTeX},
    pdfproducer={MikTeX wersja ..., pdflatex wersja ...},
    pdfkeywords={Słowa kluczowe}}

% \usepackage{showframe} % Wyświetlenie linii marginesów

\begin{document}
% \layout % Podgląd układu stron pracy
%--------------------------------------------------------
% Strona tytułowa
%--------------------------------------------------------
% Poniższe pola należy uzupełnić odpowiednimi danymi
\instytut{\{nazwa jednostki\}}
\typPracy{i} % i dla pracy inżynierskiej, m dla pracy magisterskiej
\kierunek{\{nazwa kierunku\}}
\specjalnosc{\{nazwa specjalności\}}
\title{\{Tytuł pracy\}}
\author{\{Imię i Nazwisko\}}
\album{\{liczba\}}
% Istnieje możliwość dodania drugiego autora. W tym celu należy użyć poniższej funkcji. Nie można dodać imienia autora bez podania jego numeru albumu.
% \DrugiAutor{\{Imię i nazwisko\}}
% \DrugiAlbum{\{liczba\}}
\promotor{\{tytuł/stopień naukowy, Imię i Nazwisko\}}
% Istnieje możliwość dodania opiekuna oraz konsultanta
% \opiekun{\{tytuł/stopień naukowy, Imię i Nazwisko\}}
% \konsultacje{\{tytuł/stopień naukowy, Imię i Nazwisko\}}
\date{\the\year}
\maketitle{}
%------------------------------------------------
\poltitle{
    Niepotrzebnie długi i skomplikowany tytuł pracy \\ trudny do przeczytania, zrozumienia i wymówienia
}
\engtitle{
    Unnecessarily long and complicated thesis' title \\ difficult to read, understand and pronounce
}

\cleardoublepage % Zaczynamy od nieparzystej strony
\include{tex/0-abstrakt}
\pagestyle{plain}

\cleardoublepage % Zaczynamy od nieparzystej strony
\tableofcontents % Spis treści

%------------------------------------------------
% Części pracy
%------------------------------------------------
\cleardoublepage % Powinna zaczynać się na nieparzystej stronie
\part{Analiza stanu wiedzy i dobór narzędzi}{I}
% \addcontentsline{toc}{part}{Część I.\hspace{5pt}Analiza stanu wiedzy i dobór narzędzi} 

\noindent \lipsum[1-5]
\clearpage\section{Istniejące rozwiązania}
\noindent \lipsum[2]

 % Przykładowy obrazek
\begin{figure}[!h]
    % Wyrównanie obrazka, szerokość i plik
    % Zamiast width można też użyć height, etc.
    \centering \includegraphics[width=0.5\linewidth]{logopw.png}
    % Podpis umieszczamy pod obrazkiem
    % znacznik \caption służy również do wygenerowania numeru obrazka
    
    \caption{Bardzo długi podpis.\lipsum[1]}
    % \label pozwala odwołać się do obrazka w innych miejscach za pomocą \ref
    % odwołanie \ref renderuje się jako numer obrazka,
    % dlatego zawsze najpierw używaj \caption a potem \label
    \label{fig:tradycyjne-logo-pw}
\end{figure}

% \ref wyrenderuje się jako 'Reference to image 1.1'
\lipsum[2] Reference to image \ref{fig:tradycyjne-logo-pw}.

% Przypis dolny \footnote
\lipsum[4] Lorem ipsum dolor sit amet\footnote{Lorem ipsum dolor sit amet, consectetur adipiscing elit, sed do eiusmod tempor incididunt ut labore et dolore magna aliqua. Ut enim ad minim veniam, quis nostrud exercitation ullamco laboris nisi ut aliquip ex ea commodo consequat.}, consectetur adipiscing elit.

\subsection{Przegląd urządzeń o podobnym zastosowaniu}

\noindent Poniżej znajdują się przykładowe odwołania do bibliografii. 

%Ta lista jest długa. Zmnieszamy czcionkę i odstępy.
% \begingroup

\begin{itemize} \footnotesize
    \item książka (jako całość) z minimum wymaganych pól \cite{Gorski2023},
    \item książka (jako całość) z kilkoma autorami \cite{Kurczab2016},
    \item książka (jako całość) z informacją o wydaniu oraz ISBN \cite{Rowling2000},
    \item książka (jako całość) z informacją o tłumaczu. \cite{AlKhalili2022},
    \item rodzdział książki bez wydzielonego autorstwa \cite{Smith2018},
    \item rodzdział książki bez wydzielonego autorstwa \cite{Peterson2019},
    \item rozdział książki z autorami rozdziału \cite{Kraus2010},
    \item rozdział książki z autorami rozdziału \cite{Isidori2023},
    \item rozdział książki z autorami rozdziału \cite{Gregerson2010},
    \item artykuł w czasopiśmie \cite{Nowak2023},
    \item artykuł w czasopiśmie \cite{Xu2023},
    \item artykuł w czasopiśmie \cite{Mashatan2007},
    \item artykuł w czasopiśmie \cite{Hwang2006},
    \item artykuł w czasopiśmie \cite{Joye2007},
    \item praca dyplomowa \cite{Cwik2018},
    \item praca dyplomowa \cite{Marszalek2023},
    \item praca dyplomowa \cite{Urbanski2023},
    \item praca dyplomowa \cite{Otto2004},
    \item dokumentacja techniczna \cite{LPC2212},
    \item dokumentacja techniczna \cite{NIST-SP-800-22},
    \item standard \cite{FIPS140-2},
    \item akt prawny \cite{ust2001-09-18},
    \item publikacja internetowa \cite{Schneier2017},
    \item publikacja internetowa \cite{Craddock2017},
    \item publikacja internetowa \cite{MatlabPerformance2015},
    \item materiały konferencyjne \cite{Duckworth2019},
    \item materiały konferencyjne \cite{Jun2006}.
\end{itemize}
% \endgroup

\subsection{Wady i zalety obecnych rozwiązań}
\noindent Poniżej znajdują się przykłady list. Listy punktowe używane są do wymieniania elementów, 
nieocechowanych priorytetem. Według zaleceń Rektora Politechniki Warszawskiej, w pracach
dyplomowych zalecane jest, aby wyliczenia punktowe symbolizowane były kropką
lub myślnikiem (--).

\noindent Listy punktowe -- (przykład z "kropką"):
\begin{itemize}
    \item pierwszy punkt,
    \item drugi punkt,
    \item trzeci punkt.
\end{itemize}

\noindent Listy punktowe. Przykład z „--” oraz zagnieżdżeniem:
\begin{itemize}[label=---]
    \item[--] Pierwszy punkt z myślnikiem.
    \item[--] Drugi punkt z myślnikiem.
    \item[--] Trzeci punkt z myślnikiem.
    \begin{itemize}
        \item[--] Drugi pierwszy punkt z myślnikiem.
        \item[--] Drugi drugi punkt z myślnikiem.
        \item[--] Drugi trzeci punkt z myślnikiem.
        \begin{itemize}
            \item[--] Pierwszy drugi pierwszy punkt z myślnikiem.
            \item[--] Drugi drugi drugi punkt z myślnikiem.
            \item[--] Trzeci drugi trzeci punkt z myślnikiem.
        \end{itemize}    
    \end{itemize} 
    
    \item[--] Trzeci punkt z myślnikiem.
\end{itemize}

% Lista punktowana
% Parametr label ustawia symbol punktora
\noindent Należy mieć na uwadze, że istnieje wiele różych "kresek", które wyglądają jak 
myślnik, pachną jak myślnik, ale myślnikiem nie są. Oto one:
\begin{itemize}[label=--]
    \item dywiz "-",
    \item myślnik lub półpauza "--",
    \item pauza "---"
    \item znak minus $-$
\end{itemize}   

\noindent Każdy z tych znaków ma swoje unikalne przeznaczenie. 

\noindent Przykładowa tabela
\begin{table}[H] \centering
    \caption{Opis wyprowadzeń i sygnałów}
    \label{tab:base_pins}
    \begin{tabular}{l l l l}
        \toprule[2pt]
        \textbf{\uline{Pin}}     &  \textbf{\uline{Sygnał}}&   \textbf{\uline{Typ}} &  \textbf{\uline{Przeznacznie}}             \\
        PB4                      & DCx                     & push-pull              & LCD: Wybór dane / komenda                  \\  
        PB6                      & LCD\_RES                & push-pull              & LCD: reset                                 \\ 
        PA2                      & DIM                     & PWM                    & LCD: kontrola jasności ekranu              \\ 
        PA3                      & BUZZ                    & push-pull              & Sygnał sterujący brzęczykiem               \\ 
        PA4                      & SPI\_NSS                & alternatywny           & LCD: zezwolenie na komunikację             \\ 
        PA5                      & SPI\_SCK                & alternatywny           & LCD: sygnał zegarowy                       \\ 
        PA6                      & SPI\_MISO               & alternatywny           & LCD: dane przychodządze                    \\ 
        PA7                      & SPI\_MOSI               & alternatywny           & LCD: dane wychodzące                       \\ 
        PH3-BOOT0                & ---                     & wejście                & wybór trybu wybudzenia                     \\ 
        RFI\_P                   & RFI\_P                  & alternatywny           & RF: wejście P                              \\ 
        RFI\_N                   & RFI\_N                  & alternatywny           & RF: wejście N                              \\ 
        PA10                     & LED2                    & push-pull              & sterowanie diodą 2                         \\ 
        PA11                     & SUBGHZ\_RES             & alternatywny           & RF: sygnał reset radia                     \\ 
        PA12                     & LED1                    & push-pull              & sterowanie diodą 1                         \\ 
        PA13                     & SWDIO                   & alternatywny           & Progrator: dane                            \\ 
        PA14                     & SWCLK                   & alternatywny           & Progrator: sygnał zegarowy                 \\ 
        PB0-VDD\_TCXO            & VDDTCXO                 & alternatywny           & TCXO: pin zasilający                       \\ 
        OSC\_IN                  & ---                     & alternatywny           & TCXO: sygnał zegarowy z TCXO               \\ 
        \midrule \multicolumn{3}{r}{\textbf{Suma:}} & 15,37 zł \\
        \bottomrule[2pt]
    \end{tabular}
\end{table}

\noindent Listy wyliczeniowe. Stosowane są w przypadku uporządkowanych lub upriorytetowanych elementów. Zazwyczaj symbolizujemy je liczbami arabskimi, rzymskimi czy też literami.
\begin{enumerate}
    \item Punkt pierwszy.
    \item Punkt drugi.
    \item Punkt trzeci.
\end{enumerate}
\noindent Listy definicyjne:
Używane są do objaśniania wszelkiego rodzaju pojęć,  akronimów czy też symboli, występują w dwóch
wariacjach -- z wcięciami w formie zdaniowej lub z wyrównaniem kolumnowym (w postaci tabeli).
W tym szablonie jest jednak odrębne miejsce na umieszczenie własnych akronimów i symboli. Znajduje
się ono w pliku main.tex.

\begin{description}
    \item[Kot] ma 4 nogi, miauczy i bardzo dużo śpi w ciągu dnia.
    \item[Pies] to czteronożny zwierz, który szczeka, towarzyszy człowiekowi w 
   spacerach oraz aportuje piłki.
    \item[Świnia] ma 4 nogi, kwiczy i jest zwykle zwierzęciem chlewnym, które jest 
   czasem wypowadzane na spacery.
\end{description}
\noindent Ogólne zasady tworzenia list:
\begin{itemize}
    \item Każda lista powinna zawierać co najmniej 3 elementy.
    \item Należy unikać zbyt długich wyliczeń lub wypunktowań, w takim wypadku rozważyć podzielenie 
    na mniejsze listy lub podlisty.
    \item Elementy wyliczeń, które nie przybierają postaci samodzielnych zdań rozpoczynamy małą literą.
    \item Całą listę, tj. ostatni element wyliczania, zawsze kończymy kropką.
\end{itemize}


\clearpage\section{Normy i standardy techniczne}
\noindent \lipsum[2]

    \subsection{Wymagania prawne}
    \noindent \lipsum[2]

    \subsection{Standardy projektowe}
    \noindent \lipsum[2]

\clearpage\section{Określenie problemu}
    \noindent \lipsum[2]

\clearpage\section{Dobór narzędzi}
    \noindent \lipsum[2]

    \subsection{Projekt 3D}
        \noindent \lipsum[2]

    \subsection{Schematy elektryczne}
        \noindent \lipsum[2]
        
    \subsection{Środowisko symulacyjne}
        \noindent \lipsum[2]


\clearpage\section{Wymagania funkcjonalne i techniczne}
    \noindent \lipsum[2]

    \subsection{Specyfikacja funkcjonalna}
        \noindent \lipsum[2]

    \subsection{Ograniczenia techniczne}
        \noindent \lipsum[2]
\cleardoublepage % Powinna zaczynać się na nieparzystej stronie
\part{Projekt i implementacja}{II}
\noindent \lipsum[1-3]

\clearpage\section{Koncepcja urządzenia}
    % Równanie typu 'inline':
    \noindent \lipsum[1-3] $F = m \cdot a$ lorem ipsum dolor sit amet.
    % Równanie bez numeru
    % align oznacza wyrównanie kolejnych wierszy do '&'
    % '&' służy tylko do wyrównania i nie jest renderowany
    \begin{align}
        E & = mc^2 \\
        y & = ax^2 + bx + c
    \end{align}

    \lipsum[4-5]
    % Równanie numerowane: macierze
    \begin{align}
        \begin{bmatrix}
            1 & 0 & 0 \\
            0 & 2 & 0 \\
            0 & 0 & 3
        \end{bmatrix} \cdot
        \begin{bmatrix}
            4 \\
            5 \\
            6
        \end{bmatrix} =
        \begin{bmatrix}
            4  \\
            10 \\
            18
        \end{bmatrix}
    \end{align}


    \subsection{Szkic koncepcyjny}
        \noindent \lipsum[6-7]

    \subsection{Schemat blokowy}
        \noindent \lipsum[8-9]

\clearpage\section{Dobór technologii i komponentów} 
    \noindent \lipsum[1-3]

    \subsection{Komponenty mechaniczne}
        \noindent \lipsum[4-5]

    \subsubsection{Układy elektroniczne}
        \noindent \lipsum[6-7]

    \subsubsection{Oprogramowanie sterujące} 
        \noindent \lipsum[8-9]

\clearpage\section{Symulacje i analizy projektowe}
    \subsection{Analiza wytrzymałości konstrukcji}    
        \noindent \lipsum[1-3]
    
    \subsection{Symulacja układu elektronicznego}    
        \noindent \lipsum[4-5]
    
\clearpage\section{Etapy realizacji}
    \noindent \lipsum[1-3]
    \subsection{Montaż mechaniczny}
        \noindent \lipsum[4-5]

    \subsection{Montaż elektroniczny}
        \noindent \lipsum[6-7]

    \subsection{Integracja systemu}
        \noindent \lipsum[8-9]

\clearpage\section{Dokumentacja techniczna}
    \noindent \lipsum[1-5]
\cleardoublepage % Powinna zaczynać się na nieparzystej stronie
\part{Badania i analiza powykonawcza}{III}
\noindent \lipsum[1-5]

\clearpage\section{Metodyka testowania}
\noindent \lipsum[6-8]

% Fragment kodu źródłowego programu
% \addmargin pozwala na wcięcie kodu od lewej (tu: 8mm).
% Wcięcie służy do tego, aby numery linii nie wystawały poza lewy margines.
% Druga liczba oznacza wcięcie od prawej.
\begin{addmargin}[8mm]{0mm}
\begin{lstlisting}[
    language=HTML,
    numbers=left,
    firstnumber=1,
    caption={\textbf{Hello world} w HTML},
    aboveskip=10pt
]
<html>
  <head>
    <title>Hello world!</title>
  </head>
  <body>
    Hello world!
  </body>
</html>
\end{lstlisting}
\end{addmargin}

\lipsum[9-10]

% Dla dłuższych numerów linii potrzebne jest większe wcięcie.
\begin{addmargin}[12mm]{0mm}
\begin{lstlisting}[
    language=C++,
    numbers=left,
    firstnumber=147,
    caption={Generowanie sekwencji Collatza w języku C++},
    aboveskip=10pt
]
class Collatz {
  private:
    unsigned current_val_;
    void update_val() {
        if( current_val_ % 2 == 0 )
            current_val_ /= 2;
        else
            current_val_ = current_val_ * 3 + 1;
    }

  public:
    explicit Collatz(unsigned initial_value) :
        current_val_(initial_value) {}
    void print_sequence() {
        unsigned i = 1;
        while( current_val_ > 1 ) {
            std::cout
                << "val " << i << " = " << current_val_
                << std::endl;
            update_val(); ++i;
        }
    }
};

int main() {
  // prints Collatz seqence, starting from 194375
  Collatz seq(194375);
  seq.print_sequence();
  return 0;
}
\end{lstlisting}
\end{addmargin}

\lipsum[10-12]
 
    \subsection{Testy funkcjonalne}
      \noindent\lipsum[1-3]

    \subsection{Testy wytrzymałościowe}
      \noindent\lipsum[4-7]

    \subsection{Testy wydajnościowe}
      \noindent\lipsum[8-10]

\clearpage\section{Wyniki testów}
  \noindent\lipsum[1-7]

    \subsection{Porównanie wyników z założeniami}
      \noindent\lipsum[1-3]

    \subsection{Analiza błędów}
      \noindent\lipsum[4-6]

\clearpage\section{Analiza ekonomiczna}  
  \noindent\lipsum[1-7]

  \subsection{Koszty materiałów i komponentów}
    \noindent\lipsum[1-3]

  \subsection{Koszty produkcji seryjnej}
    \noindent\lipsum[4-7]

  \subsection{Porównanie kosztów z alternatywami}
    \noindent\lipsum[8-10]

\clearpage\section{Problemy napotkane podczas budowy i ich rozwiązanie}
\noindent \lipsum[1-5]

\clearpage\section*{Podsumowanie}
    \noindent \lipsum[1-9]
%------------------------------------------------
\cleardoublepage % Powinna zaczynać się na nieparzystej stronie
\printbibliography % Bibliografia
\clearpage

% Wykaz symboli i skrótów.
% Należy posortować symbole alfabetycznie
% we własnym zakresie. Makro \acronymlist
% generuje właściwy tytuł sekcji w zależności od języka.
% Makro \acronym dodaje skrót/symbol do listy,
% zapewniając podstawowe formatowanie.
\acronymlist
\acronym{EiTI}{Wydział Elektroniki i Technik Informacyjnych}
\acronym{PW}{Politechnika Warszawska}
\acronym{WEIRD}{ang. \emph{Western, Educated, Industrialized, Rich and Democratic}}
\vspace{0.8cm}

%--------------------------------------
% Spisy
%--------------------------------------
\pagestyle{plain}
\newpage   
\listoffigurestoc       % Spis rysunków
\newpage                % Odstęp pionowy
\listoftablestoc        % Spis tabel
\newpage                % Odstęp pionowy
\listofappendicestoc    % Spis załączników

%--------------------------------------
% Załączniki
%--------------------------------------
% Ilustracje i tabele w załącznikach nie pojawią się
% w głównym spisie treści
\captionsetup[figure]{list=no}
\captionsetup[table]{list=no}
\clearpage
\appendix{Nazwa załącznika A}
\noindent\lipsum[1-3]
\begin{figure}[!h]
	\centering \includegraphics[width=0.5\linewidth]{logopw2.png}
	\caption{Ilustracja, która nie znajdzie się w spisie treści}
\end{figure}
\lipsum[4-7]

\clearpage
\appendix{Nazwa załącznika B}
\lipsum[1-2]
\begin{table}[H] \centering
    \caption{Tabela, która nie znajdzie się w spisie treści}
    \label{tab:tab}
    \begin{tabular}{c c c c c}
        \toprule[1pt]
        % Należy starać się, aby podkreślenie pod nagłówkiem dwukolumnowym objęło obie te kolumny
         & \multicolumn{2}{c}{\textbf{\uline{\hspace{0.5cm} Cell \hspace{0.5cm}}}} & \multicolumn{2}{c}{\textbf{\uline{\hspace{0.5cm} Cell \hspace{0.5cm}}}}   \\
        \textbf{{Cell}}  &\textbf{{Cell}}     &  \textbf{Cell}&   \textbf{Cell} &  \textbf{Cell} \\  \midrule
        cell  & cell & cell & cell & cell   \\
        cell  & cell & cell & cell & cell   \\
        cell  & cell & cell & cell & cell   \\
        cell  & cell & cell & cell & cell   \\
        cell  & cell & cell & cell & cell   \\ 
        cell  & cell & cell & cell & cell   \\ \midrule
        \multicolumn{4}{r}{\textbf{Cell}} & Cell \\
        \bottomrule[1pt]
    \end{tabular}
\end{table}
\lipsum[3-4]



\end{document}
